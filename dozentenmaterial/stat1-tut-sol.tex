\documentclass[11pt,a4paper]{article}

\usepackage[utf8]{inputenc}

\usepackage{xunicode}
\usepackage{xltxtra}
\usepackage{polyglossia}
\setdefaultlanguage{german}
\usepackage{lmodern}
\usepackage{csquotes}

%Einstellung Seitenränder
\usepackage[left=2cm,right=2cm,top=2cm,bottom=2cm]{geometry}

% math-umgebung
\usepackage{amsmath}
\usepackage{amsfonts}
\usepackage{amssymb}

% Color und Hyperlink packages
\usepackage{hyperref}
\usepackage[svgnames,hyperref]{xcolor}

% Datumspaket
\usepackage[german]{isodate}

% Table packages
\usepackage{tabularx}
\newcolumntype{L}[1]{>{\raggedright\arraybackslash}p{#1}} % linksbündig mit Breitenangabe
\newcolumntype{C}[1]{>{\centering\arraybackslash}p{#1}} % zentriert mit Breitenangabe
\newcolumntype{R}[1]{>{\raggedleft\arraybackslash}p{#1}} \usepackage{booktabs}
\usepackage{longtable}
\usepackage{multirow}

% Code highlighting
\usepackage{color}
\newcommand{\correct}[1]{\textcolor{ForestGreen}{#1}}
\definecolor{dkgreen}{rgb}{0,0.6,0}
\definecolor{gray}{rgb}{0.5,0.5,0.5}
\definecolor{mauve}{rgb}{0.58,0,0.82}
	
% Fortlaufende Zählung
\usepackage{enumitem}

%Einstellungen hyperlink
\hypersetup{
    colorlinks=true,
    filecolor=Purple,    
    linkcolor = mauve,  
    urlcolor=MediumSeaGreen,
    citecolor = black,
    pdftitle={Seminarplan}
}

\urlstyle{same}
% how to use Hyperlinks: https://de.overleaf.com/learn/latex/Hyperlinks

%New Font
\usepackage{fontspec} % changing font (unten freimachen für Änderung
	% \defaultfontfeatures{Mapping = tex-text}
	\setmainfont{Fira Sans} %user-defined Font.

% Setting math font	
\usepackage{unicode-math}
	\setmathfont{Fira Math}
	
% Name, Titel, etc.
\author{Serdar Selova, Alexander Lechner \\ \TeX: B. Philipp Kleer \\ \footnotesize Team Professur für Methoden der Politikwissenschaft}
\title{%
  Tutorien (Lösungen) \\
  \normalsize Statistik I \\
%  WiSe/SoSe
  }

\date{\today \\ \small{Version: v2.1}}

\begin{document}
\maketitle

Dieses Material ist ausschließlich für Tutor:innen der Professur für Methoden der Politikwissenschaft unter Berücksichtigung der Demokratie- und politischen Sozialisationsforschung. \textbf{Eine Weitergabe ist untersagt!}

\textbf{Hinweis:} Da das \textit{typesetting} der Formelsammlung mit englischem Sprachformat erfolgt, ist das Dezimalzeichen ein Punkt ($.$) und nicht ein Komma (wie im Deutschen üblich).

\tableofcontents 

\newpage

\input{subfiles/stat1-tut1-sol}

\newpage

\input{subfiles/stat1-tut2-sol}

\newpage

\input{subfiles/stat1-tut3-sol}

\newpage

\input{subfiles/stat1-tut4-sol}

\newpage

\input{subfiles/stat1-tut5-sol}

\newpage

\input{subfiles/stat1-tut6-sol}

\newpage

\input{subfiles/stat1-tut7-sol}

\newpage

\input{subfiles/stat1-tut8-sol}

\newpage

\input{subfiles/stat1-tut9-sol}

\end{document}

